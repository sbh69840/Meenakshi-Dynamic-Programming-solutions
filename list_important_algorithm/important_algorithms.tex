\documentclass[12pt]{article}
\usepackage{amsmath, amssymb, amsthm, amsfonts}
\usepackage{fullpage}
\usepackage{times}
\usepackage{tikz}
\usepackage{caption}
\usetikzlibrary{arrows}
\usepackage{verbatim}
\usepackage{array}
%\setlength{\extrarowheight}{1pt}
\usepackage{booktabs} %for top, middle and bottomline
\usepackage{bigstrut}
\usepackage{tcolorbox}
\setlength\bigstrutjot{3pt}
\usepackage{verbatim}
\usepackage{mathtools}


\usepackage{graphicx}


\usepackage[colorlinks = true,
linkcolor = red,
urlcolor  = red,
citecolor = red,
anchorcolor = red]{hyperref}
\usepackage[display]{texpower}
%\usepackage[screen,nopanel]{pdfscreen}
%\usepackage{booktabs}
\usepackage{lmodern}
\makeatletter
\newlength\mylena
\newlength\mylenb
\newcommand\mystrut[1][2]{%
	\setlength\mylena{#1\ht\@arstrutbox}%
	\setlength\mylenb{#1\dp\@arstrutbox}%
	\rule[\mylenb]{0pt}{\mylena}}
\makeatother

\newtcolorbox{mybox}[3][]
{
	colframe = #2!25,
	colback  = #2!10,
	coltitle = #2!20!black,
	title    = #3,
	#1,
}

\definecolor{grannysmithapple}{rgb}{0.66, 0.89, 0.63}
\definecolor{goldenyellow}{rgb}{1.0, 0.87, 0.0}
\definecolor{electricindigo}{rgb}{0.44, 0.0, 1.0}

% You can add more of these if it is helpful.
\newtheorem{theorem}{Theorem}
\newtheorem{definition}{Definition}
\newtheorem{lemma}{Lemma}
\newtheorem{corollary}{Corollary}
\renewcommand{\qed}{\hfill $\framebox(6,6){}$}
\newtheorem*{prob*}{Problem statement}
\newtheorem{prob}{Problem statement:}[section]
\newtheorem{ques}{Question:}[section]
\newtheorem{ex}{EXAMPLE:}[section]
\newcommand*\xor{\mathbin{\oplus}}
\newcommand{\dint}{\displaystyle\int}
%\newcommand{\dfrac}{\displaystyle\frac}
\newtheorem{remark}{Remark}
\definecolor{agreen}{RGB}{102,102,102}
% Use the proof environment when the proof immediately follows the corresponding
% theorem or lemma.
\renewenvironment{proof}{\par{\noindent \bf Proof:}}{\qed \par}

% Use the proofof environment when the proof appears later.
\newenvironment{proofof}[1]{\par{\noindent \bf Proof of #1:}}{\qed\par}

% Use the proofsketch environment for less formal proof ideas.
\newenvironment{proofsketch}{\par{\noindent \bf Proof Sketch:}}{\qed\par}
\setlength{\parindent}{0cm}

% CHANGE THESE DEFINITIONS AS APPROPRIATE:
\def \scribe {Shivaraj B H} % Change this to your names
\def \lecturer {} % Change this only if there is a guest lecturer
\def \lecturedate {}  % Change this to the date of class
\def \lecturenumber {} % Change this to the number of the class
\def \lecturetitle {} % Change this too
\begin{document}
	\title{
		\fontsize{16}{24}\bfseries
		\makebox[\textwidth][s]{
			\makebox[-10pt][l]{
				}
			\hfill
			\begin{tabular}[b]{@{}c@{}}
			    Algo Bucket
			\end{tabular}%
			\hfill
		}
	}
	\date{\today}
	\maketitle



	
	\begin{mybox}{goldenyellow}{}
		\begin{prob*}		
			Here I will be listing all the important algorithm's that you have to remember before going to any interview or coding competitions. (Not necessary that I mention every single one of them)
			\end{prob*}
	\end{mybox}


\vspace{0.3cm}				
			
\begin{mybox}{electricindigo}{}
	\textbf{Solution :} 
	\begin{itemize}
		\item[\textbf{1}] Kadane's algorithm 
		\item[\textbf{2}] Johnson and Trotter algorithm (Permutations of a string using mobile integers) 
		\item[\textbf{3}] Euclidean algorithm (To find GCD) 
		\item[\textbf{4}] \href{https://en.wikipedia.org/wiki/Extended_Euclidean_algorithm}{Extended Euclidean algorithm} (Find coefficients of Bezout's identity eg: find x,y in ax+by=gcd(a,b))
		\item[\textbf{5}] Matrix exponentiation (To find solutions that consists of finding something similar to fibonacci numbers, (Where you add previous numbers to get current number))
		\item[\textbf{5}] Partition problem (To find ways to split )
		
	\end{itemize}
	

\end{mybox}


	

		
		

\end{document} 
